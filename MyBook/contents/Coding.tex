\chapter{Linux基础与编程知识的学习}

很多同学在学习生物信息学的开始往往顺着别人的Pipeline直接就做下来了,得到了结果但是对于流程控制的细节和原理知之甚少。 
“磨刀不误砍柴工”,在读本书之前,我建议你先拥有良好的编程技能和数据分析技能,再进行一定的生物学学习,最后再进行分析。

关于编程,Linux命令行是我们进行所有操作的基石,一定要学扎实,我在学习Linux的时候读过《鸟哥的Linux私房菜(基础篇)》这本书,鸟哥讲的很琐碎不过你要尽量学习过,至少过一遍,再去学习其他编程语言,这样你会更好地理解计算机和编程。 
可以租用服务器提供商的VPS搭建自己的个人博客这种方式来练习你的Linux基础技能,作者就是通过搭建网站的方式对Linux系统和命令行以及网络通讯有了一定的了解,这对接下来的学习非常重要。
当你拥有了一定的Linux使用经验之后,就会发现Bash命令行的局限性很多,语法也比较混乱,但是在简单的字符串处理上Bash脚本是效率非常高和方便易用的。当然你可能不知道什么叫Bash(或者什么是Shell),去百度或者Google自行学习。
在学习生物信息学的路上,最好的老师就是搜索引擎,希望你能牢牢记住这一点,尽量在求助别人之前先自己搜索一下,研究研究。 

我们谈到了Bash脚本的局限性,简单的字符串处理是它的强项,但是数据一旦开始复杂,Bash的使用就没有那么方便了,这时候我们就必须要掌握一到两门编程语言,作为一名即将进入生物信息学领域的研究人员,我强烈建议你熟练掌握Python和R两种编程语言,业内人员用的最多的就是Python和R,也有很多现成的程序,扩展包,算法实现和问题案例。 当然也可以学习Java,Julia,Perl和Go,但是会走不少弯路。 目前来说,Python的灵活和简单使得我们能够轻易实现想法,搭建分析框架(如Snakemake),得到计算结果,数据处理(Pandas)。 
而R是我们高效地进行统计学分析和数据可视化分析绕不开的好工具,R的Bioconductor包在生物信息学中占据了很重要的地位,R是必须要掌握的基本技能。

总结一下,学习本书之前,我希望你的编程能力达到这种水平。


\begin{itemize}
	\item 
	第一,掌握Linux的基本使用,环境配置,达到能简单搭建个人博客水平(比如使用Markdown标记语言结合Nginx与Hexo。
	\item 第二,Python达到熟练进行数据分析的水平,掌握基本语法之后熟练运用os,pandas,numpy扩展包,对matplotlib合seaborn有一定的了解。
	\item 
	第三,R,掌握基本语法和向量、数组操作之后,尽量能够熟练操作frame,对ggplot2等扩展包有一定了解。 以上这些都是本书不会提起到的内容,但是默认你已经基本达到了上述水平,补充一句,不会就百度或Google,这真的很重要。
\end{itemize}

写书是一件非常复杂的事情,Linux、Python、R的基础知识,甚至Python中一个第三方包(如Pandas)的学习,都够写厚厚一本书的。我不打算在本书中详细展开上述知识,在用到什么的时候我就会稍微解释几句,如果看的不是太懂尽量去搜索引擎检索一下,往往都能找到答案。
我会列出一些学习资源,希望你都能去学一学,多动手敲代码,多动脑思考,多尝试,如果你没有基础,你可以参考我列出的书目和多媒体资源,按照下面的顺序去学,如果你有一定基础,可以选择性的再看看:

\begin{itemize}
	\item
	首先一定要看的是,《鸟哥的Linux私房菜》\footnote{http://linux.vbird.org/},力求掌握Shell脚本编程的内容,了解书中阐述的计算机基础知识,权限操作,如果时间充裕,尽量系统地看完。
	\item
	接下来是一个新手友好的视频资源,《懂中文就会,黑马程序员Python基础视频教程》\footnote{https://www.bilibili.com/video/BV1ex411x7Em?from=search\&seid=5478123447193797430},内容设置非常好,建议跟着一起练习一遍。通过这个视频你可以掌握Python的基本应用和简单的引用第三方包。
	\item
	接下来去学习另一个新手友好的视频资源,需要付费。不过小钱,买吧!《R语言入门基础》\footnote{https://edu.csdn.net/course/detail/24913},这个讲的比较细致,当然也有点啰嗦,R入门看这个就可以了。
\end{itemize}

上面几个资源是我认为你必须要牢牢掌握的知识,所以别犹豫,努力去学习吧!

接下来,是我们作为生信工作者必须要掌握的数据分析技能,你可以选择Python和R其中之一先学好,然后就可以尝试进行一些分析了,但是最终你还是要把两者全部掌握才行:

\begin{itemize}
	\item 
	《Python Data Science Handbook》,中文版叫《Python数据科学手册》,O'Reilly Media, Inc.出版的质量非常不错的Python数据分析教程书籍。
	\item
	《R for Data Science》,中文版叫《R数据科学》,相对的,R我也推荐O'Reilly Media, Inc.出版的这本。
\end{itemize}

经过上面两本书的熏陶,相信你可以胜任一定的生物信息学分析工作了,你可以跟着本书一点点去学习如何展开一项分析Project,如何搭建属于你自己的固定分析流程来简化工作。

不要止步于此,这里有一些好的资源,也希望你去好好看看:

\begin{itemize}
	\item 
《生信技能树-生信人应该这样学R语言》\footnote{https://www.bilibili.com/video/BV1cs411j75B?p=1},这个前面R基础讲的很乱,没有一定的R基础基本听不懂,但是结合了生物信息学特点去讲解,也有不少干货,后面还不错,所以建议先看上面的R语言入门基础,再来看这个课程,有耐心争取看完, 听不下去也无所谓。
\end{itemize}


