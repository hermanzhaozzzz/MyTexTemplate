\chapter{Kaggle简介}
Kaggle成立于2010年,是一个进行数据发掘和预测竞赛的在线平台。从公司的角度来讲,可以提供一些数据,进而提出一个实际需要解决的问题;从参赛者的角度来讲,他们将组队参与项目,针对其中一个问题提出解决方案,最终由公司选出的最佳方案可以获得5K-10K美金的奖金\cite{A2Mia}。

除此之外,Kaggle官方每年还会举办一次大规模的竞赛,奖金高达一百万美金,吸引了广大的数据科学爱好者参与其中。从某种角度来讲,大家可以把它理解为一个众包平台,类似国内的猪八戒。但是不同于传统的低层次劳动力需求,Kaggle一直致力于解决业界难题,因此也创造了一种全新的劳动力市场——不再以学历和工作经验作为唯一的人才评判标准,而是着眼于个人技能,为顶尖人才和公司之间搭建了一座桥梁\cite{A2Mia}。

那么,为什么要在这本书里提及Kaggle呢?
我们在前面的章节中介绍和学习了使用Python和R两种语言,同时我们学习了IDE——Jupyter,生物信息学中的很多分析流程,最后都会生成一些表格来反应各种信息,最简单的例子是使用Cufflinks计算基因的相对表达值,然后进行基因的差异表达分析,相对复杂一些的如GATK进行SNP-calling,统计碱基的突变信息,我们最终会拿到类似Excel表格的数据形式。
接下来我们的问题是如何从这些数据表格中得到有价值的信息。
当然我们可以使用Excel进行简单的数据处理和绘图工作,但在生物信息学的工作环境下,大部分时候Excel不能满足我们的分析需求,我们可以使用Python中的Pandas模块,或者R语言来做更加个性化的,性能更高的分析工作,以及完成图表绘制。

我给大家的建议是选择Python和R相关的Kaggle项目,先学习如何进行简单的数据清洗,然后学习如何进行简单的数据分析统计工作,学会向量化运算方法和将自定义函数“map/apply”到你的向量化数据中进行数据的变换,简单的四则运算,最后学习一些简单的机器学习方法来应用在手上的数据上。在每个分析的工作中,尽量进行数据的可视化工作。如果你能过在Kaggle的比赛中取得不错的成绩,将来即使不从事生物信息学研究,你的Kaggle成绩也能在数据分析相关行业中有不错的认可度,在求职中提供一定的竞争力。

在经过Kaggle项目的练习之后,你就可以摆脱仅仅会用一些生物信息学软件来生成数据了,你将能够有一定的个性化分析能力,这在组学研究中比较重要,有了这些DIY分析意识,你就有可能在组学数据中发现新的idea,新的规律,增加你的生物信息学上限。有经验的从业者都知道,决定生物信息学分析下线的,是你的编程能力,而生物学、化学知识,以及统计学知识,决定了生物信息学分析的上限。达到了一定高度的生物信息学研究者,都在尝试各种先进的算法思想和统计学知识。Kaggle正是你踏入算法和统计大门的敲门砖。