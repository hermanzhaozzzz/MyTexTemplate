\documentclass[cn,11pt,chinese]{elegantbook}

\title{生物信息学}
\subtitle{使用Python与R进行生物信息学分析}

\author{赵华男}
%\institute{}
%\date{2020-05-01}
%\version{1}
%\bioinfo{自定义}{信息}

\extrainfo{生物信息学保姆级教程}

\logo{logo-blue.png}
\cover{cover.jpg}

% 本文档命令
\usepackage{array}
\newcommand{\ccr}[1]{\makecell{{\color{#1}\rule{1cm}{1cm}}}}

\begin{document}

\maketitle
\frontmatter

% author, introduction
\chapter*{关于作者}
赵华男(1995年生),山东省滕州人,本科毕业于西北农林科技大学动物医学专业,2019年进入清华大学生命科学学院攻读博士学位,培养单位为PTN项目,实验室位于北京大学生命科学学院金光生命科学大楼,博士期间的研究方向是关于Crispr cas9的基因编辑相关技术。


\chapter*{前言}
在写这本书的一开始,我还在读博士一年级,因为自己也算一个从零开始从事生物信息学研究的案例,所以也有很多东西需要去学习,最开始的时候我通过写\href{http://zhaohuanan.cc}{博客}的方式进行学习和记录,后来觉得这样子不够系统,写的也比较随意,免不了一些错误,于是怀着忐忑的心情,我最终决定还是要写这么一本书,一方面,希望自己能够严谨、系统地完成生物信息学的博士阶段学习,并且通过写书来记录自己的学习历程,还可以时不时回来翻看和修改,另一方面,希望能在将来的某一天,在我认为这本书已经相对成熟的时候出版,将我自己的学习历程分享给大家,也能给各位即将踏入生物信息学研究领域的读者提供一些帮助。

\begin{center}
  在此感谢我的师兄\href{https://www.zhihu.com/people/meng_howard}{孟浩巍}对我的指导,师兄的指导让我少走了很多弯路,也对生物信息学有了更深刻的认识。
\end{center}

希望未来的某天,能有更多话写在这个位置,给自己一个鼓励:Go!


\vskip 1.5cm

\begin{flushright}
赵华男\\
2020-06-01\\
北京
\end{flushright}


% 目录
\tableofcontents

% 正文
\mainmatter
% 第一章 引言
\chapter{引言}

\section{什么是生物信息学}
\input{contents/引言/什么是生物信息学}
\section{本书的组织架构}
\input{contents/引言/本书的组织架构}
\section{对读者的建议}
\input{contents/引言/对读者的建议}

% 第二章 Linux基础与编程知识的学习
\chapter{Linux基础与编程知识的学习}

很多同学在学习生物信息学的开始往往顺着别人的Pipeline直接就做下来了,得到了结果但是对于流程控制的细节和原理知之甚少。 
“磨刀不误砍柴工”,在读本书之前,我建议你先拥有良好的编程技能和数据分析技能,再进行一定的生物学学习,最后再进行分析。

关于编程,Linux命令行是我们进行所有操作的基石,一定要学扎实,我在学习Linux的时候读过《鸟哥的Linux私房菜(基础篇)》这本书,鸟哥讲的很琐碎不过你要尽量学习过,至少过一遍,再去学习其他编程语言,这样你会更好地理解计算机和编程。 
可以租用服务器提供商的VPS搭建自己的个人博客这种方式来练习你的Linux基础技能,作者就是通过搭建网站的方式对Linux系统和命令行以及网络通讯有了一定的了解,这对接下来的学习非常重要。
当你拥有了一定的Linux使用经验之后,就会发现Bash命令行的局限性很多,语法也比较混乱,但是在简单的字符串处理上Bash脚本是效率非常高和方便易用的。当然你可能不知道什么叫Bash(或者什么是Shell),去百度或者Google自行学习。
在学习生物信息学的路上,最好的老师就是搜索引擎,希望你能牢牢记住这一点,尽量在求助别人之前先自己搜索一下,研究研究。 

我们谈到了Bash脚本的局限性,简单的字符串处理是它的强项,但是数据一旦开始复杂,Bash的使用就没有那么方便了,这时候我们就必须要掌握一到两门编程语言,作为一名即将进入生物信息学领域的研究人员,我强烈建议你熟练掌握Python和R两种编程语言,业内人员用的最多的就是Python和R,也有很多现成的程序,扩展包,算法实现和问题案例。 当然也可以学习Java,Julia,Perl和Go,但是会走不少弯路。 目前来说,Python的灵活和简单使得我们能够轻易实现想法,搭建分析框架(如Snakemake),得到计算结果,数据处理(Pandas)。 
而R是我们高效地进行统计学分析和数据可视化分析绕不开的好工具,R的Bioconductor包在生物信息学中占据了很重要的地位,R是必须要掌握的基本技能。

总结一下,学习本书之前,我希望你的编程能力达到这种水平。


\begin{itemize}
	\item 
	第一,掌握Linux的基本使用,环境配置,达到能简单搭建个人博客水平(比如使用Markdown标记语言结合Nginx与Hexo。
	\item 第二,Python达到熟练进行数据分析的水平,掌握基本语法之后熟练运用os,pandas,numpy扩展包,对matplotlib合seaborn有一定的了解。
	\item 
	第三,R,掌握基本语法和向量、数组操作之后,尽量能够熟练操作frame,对ggplot2等扩展包有一定了解。 以上这些都是本书不会提起到的内容,但是默认你已经基本达到了上述水平,补充一句,不会就百度或Google,这真的很重要。
\end{itemize}

写书是一件非常复杂的事情,Linux、Python、R的基础知识,甚至Python中一个第三方包(如Pandas)的学习,都够写厚厚一本书的。我不打算在本书中详细展开上述知识,在用到什么的时候我就会稍微解释几句,如果看的不是太懂尽量去搜索引擎检索一下,往往都能找到答案。
我会列出一些学习资源,希望你都能去学一学,多动手敲代码,多动脑思考,多尝试,如果你没有基础,你可以参考我列出的书目和多媒体资源,按照下面的顺序去学,如果你有一定基础,可以选择性的再看看:

\begin{itemize}
	\item
	首先一定要看的是,《鸟哥的Linux私房菜》\footnote{http://linux.vbird.org/},力求掌握Shell脚本编程的内容,了解书中阐述的计算机基础知识,权限操作,如果时间充裕,尽量系统地看完。
	\item
	接下来是一个新手友好的视频资源,《懂中文就会,黑马程序员Python基础视频教程》\footnote{https://www.bilibili.com/video/BV1ex411x7Em?from=search\&seid=5478123447193797430},内容设置非常好,建议跟着一起练习一遍。通过这个视频你可以掌握Python的基本应用和简单的引用第三方包。
	\item
	接下来去学习另一个新手友好的视频资源,需要付费。不过小钱,买吧!《R语言入门基础》\footnote{https://edu.csdn.net/course/detail/24913},这个讲的比较细致,当然也有点啰嗦,R入门看这个就可以了。
\end{itemize}

上面几个资源是我认为你必须要牢牢掌握的知识,所以别犹豫,努力去学习吧!

接下来,是我们作为生信工作者必须要掌握的数据分析技能,你可以选择Python和R其中之一先学好,然后就可以尝试进行一些分析了,但是最终你还是要把两者全部掌握才行:

\begin{itemize}
	\item 
	《Python Data Science Handbook》,中文版叫《Python数据科学手册》,O'Reilly Media, Inc.出版的质量非常不错的Python数据分析教程书籍。
	\item
	《R for Data Science》,中文版叫《R数据科学》,相对的,R我也推荐O'Reilly Media, Inc.出版的这本。
\end{itemize}

经过上面两本书的熏陶,相信你可以胜任一定的生物信息学分析工作了,你可以跟着本书一点点去学习如何展开一项分析Project,如何搭建属于你自己的固定分析流程来简化工作。

不要止步于此,这里有一些好的资源,也希望你去好好看看:

\begin{itemize}
	\item 
《生信技能树-生信人应该这样学R语言》\footnote{https://www.bilibili.com/video/BV1cs411j75B?p=1},这个前面R基础讲的很乱,没有一定的R基础基本听不懂,但是结合了生物信息学特点去讲解,也有不少干货,后面还不错,所以建议先看上面的R语言入门基础,再来看这个课程,有耐心争取看完, 听不下去也无所谓。
\end{itemize}



%测序原理
%常见标准文件格式
%质量控制
%常用mapping软件及其核心算法
%两大神器——从现在开始让分析流程化

% 通过Snakemake与Jupyter进行生物信息学分析——实践出真知
\chapter{通过Snakemake与Jupyter进行生物信息学分析——实践出真知}
%\section{NGS的质量控制标准流程}
%\input{contents/Workflows/NGS的质量控制标准流程}
%
%\section{NGS的DNA-mapping标准流程}
%\input{contents/Workflows/NGS的DNA-mapping标准流程}
%
%
%\section{NGS的mRNA-mapping标准流程}
%\input{contents/Workflows/NGS的mRNA-mapping标准流}
%
%\section{mRNA差异表达分析参考流程}
%\input{contents/Workflows/mRNA差异表达分析参考流程}
%
%\section{Alternative Splicing分析参考流程}
%\input{contents/Workflows/Alternative Splicing分析参考流程}
%
%\section{DNA GATK SNP-calling分析参考流程}
%\input{contents/Workflows/DNA GATK SNP-calling分析参考流程}
%
%\section{RNA GATK SNP-calling分析参考流程}
%\input{contents/Workflows/RNA GATK SNP-calling分析参考流程}
%
%\section{ATAC-seq分析参考流程}
%\input{contents/Workflows/ATAC-seq分析参考流程}
%
%\section{ChIP-seq分析参考流程}
%\input{contents/Workflows/ChIP-seq分析参考流程}
%
%\section{Detect-seq分析参考流程}
%\input{contents/Workflows/Detect-seq分析参考流程}

% 机器学习在生物信息学中的应用
\chapter{机器学习——机器学习知识体系介绍}
\section{机器学习与规则编写程序的区别}
\section{机器学习原理图}
\section{机器学习的学习路径和方法}
\section{机器学习需要哪些知识}
\section{概率统计与机器学习之间的关系}
\section{推荐参考书籍}



%核心算法的理解
%统计知识
%重复几篇顶级论文的核心结论
%杂七杂八的笔记
\chapter{Kaggle简介}
Kaggle成立于2010年,是一个进行数据发掘和预测竞赛的在线平台。从公司的角度来讲,可以提供一些数据,进而提出一个实际需要解决的问题;从参赛者的角度来讲,他们将组队参与项目,针对其中一个问题提出解决方案,最终由公司选出的最佳方案可以获得5K-10K美金的奖金\cite{A2Mia}。

除此之外,Kaggle官方每年还会举办一次大规模的竞赛,奖金高达一百万美金,吸引了广大的数据科学爱好者参与其中。从某种角度来讲,大家可以把它理解为一个众包平台,类似国内的猪八戒。但是不同于传统的低层次劳动力需求,Kaggle一直致力于解决业界难题,因此也创造了一种全新的劳动力市场——不再以学历和工作经验作为唯一的人才评判标准,而是着眼于个人技能,为顶尖人才和公司之间搭建了一座桥梁\cite{A2Mia}。

那么,为什么要在这本书里提及Kaggle呢?
我们在前面的章节中介绍和学习了使用Python和R两种语言,同时我们学习了IDE——Jupyter,生物信息学中的很多分析流程,最后都会生成一些表格来反应各种信息,最简单的例子是使用Cufflinks计算基因的相对表达值,然后进行基因的差异表达分析,相对复杂一些的如GATK进行SNP-calling,统计碱基的突变信息,我们最终会拿到类似Excel表格的数据形式。
接下来我们的问题是如何从这些数据表格中得到有价值的信息。
当然我们可以使用Excel进行简单的数据处理和绘图工作,但在生物信息学的工作环境下,大部分时候Excel不能满足我们的分析需求,我们可以使用Python中的Pandas模块,或者R语言来做更加个性化的,性能更高的分析工作,以及完成图表绘制。

我给大家的建议是选择Python和R相关的Kaggle项目,先学习如何进行简单的数据清洗,然后学习如何进行简单的数据分析统计工作,学会向量化运算方法和将自定义函数“map/apply”到你的向量化数据中进行数据的变换,简单的四则运算,最后学习一些简单的机器学习方法来应用在手上的数据上。在每个分析的工作中,尽量进行数据的可视化工作。如果你能过在Kaggle的比赛中取得不错的成绩,将来即使不从事生物信息学研究,你的Kaggle成绩也能在数据分析相关行业中有不错的认可度,在求职中提供一定的竞争力。

在经过Kaggle项目的练习之后,你就可以摆脱仅仅会用一些生物信息学软件来生成数据了,你将能够有一定的个性化分析能力,这在组学研究中比较重要,有了这些DIY分析意识,你就有可能在组学数据中发现新的idea,新的规律,增加你的生物信息学上限。有经验的从业者都知道,决定生物信息学分析下线的,是你的编程能力,而生物学、化学知识,以及统计学知识,决定了生物信息学分析的上限。达到了一定高度的生物信息学研究者,都在尝试各种先进的算法思想和统计学知识。Kaggle正是你踏入算法和统计大门的敲门砖。


%\nocite{*} 
\bibliography{bib/reference,bib/library,bib/papers}
\appendix
\include{contents/appendix01}



\end{document}
